\usepackage{amsmath}%
\usepackage{amssymb}%
\usepackage{amsfonts}
\usepackage{amsthm}

\usepackage{enumerate}
\usepackage{booktabs}
\usepackage{nccmath}
\usepackage{diagbox}
\usepackage{url}
\usepackage{xspace}
\usepackage{mathbbol}
\usepackage{comment}
\usepackage{braket}
\usepackage{mathptmx}

% graphics
\usepackage{tikz}
\usepackage{pgf}
\usepackage{graphicx}
\usepackage{pgfplots}
\pgfplotsset{compat=1.14}
\usepackage{pgfplotstable}
\usetikzlibrary{patterns}
\usetikzlibrary{external}
\usepackage{epsfig}
\usepackage{wrapfig}
\usepackage{color}
\usepackage{floatflt}

\usetikzlibrary{arrows}


% copied from hera paper - check
%\usepackage{subcaption}

%\newtheorem{theorem}{Theorem}
%\newtheorem{lemma}[theorem]{Lemma}
%\newtheorem{proposition}[theorem]{Proposition}
%\newtheorem{corollary}[theorem]{Corollary}
%\newtheorem{definition}[theorem]{Definition}


% copied from expensive distances paper
%
\usepackage{algorithm}
\usepackage[noend]{algpseudocode}




% disable ugly latex defaults
\renewcommand{\leq}{\leqslant}
\renewcommand{\geq}{\geqslant}
\renewcommand{\epsilon}{\varepsilon}
\renewcommand{\phi}{\varphi}
\newcommand{\eps}{\varepsilon}

% hyphen in math mode
\mathchardef\mhyphen="2D

% function restriction
\newcommand\restr[2]{{% we make the whole thing an ordinary symbol
  \left.\kern-\nulldelimiterspace % automatically resize the bar with \right
  #1 % the function
  \vphantom{\big|} % pretend it's a little taller at normal size
  \right|_{#2} % this is the delimiter
  }}

% natural transformation arrow
\newcommand{\naturalto}{%
  \mathrel{\vbox{\offinterlineskip
    \mathsurround=0pt
    \ialign{\hfil##\hfil\cr
      \normalfont\scalebox{1.2}{.}\cr
%      \noalign{\kern-.05ex}
      $\longrightarrow$\cr}
  }}%
}

% do not use DeclareMathOperator - subscript issue
\newcommand{\Hom}{\mathrm{Hom}}

% to not annoy me
\newcommand{\Alpha}{A}
\newcommand{\Beta}{B}
\newcommand{\Epsilon}{E}

% common rings/fields/etc
\newcommand{\NN}{\mathbb{N}}
\newcommand{\ZZ}{\mathbb{Z}}
\newcommand{\ZnZ}{\mathbb{Z}/n\mathbb{Z}}
\newcommand{\QQ}{\mathbb{Q}}
\newcommand{\RR}{\mathbb{R}}
\newcommand{\CC}{\mathbb{C}}
\newcommand{\HH}{\mathbb{H}}
\newcommand{\Fp}{\mathbb{F}_p}

\newcommand{\hence}{\implies}
\newcommand{\ergo}{\implies}

% gothic letters
\newcommand{\pig}{\mathfrak{p}}
\newcommand{\aig}{\mathfrak{a}}
\newcommand{\qig}{\mathfrak{q}}
\newcommand{\mig}{\mathfrak{m}}
\newcommand{\nilrad}{\mathfrak{N}}

\DeclareMathOperator*{\argmin}{arg\,min}
\DeclareMathOperator*{\argmax}{arg\,max}
\DeclareMathOperator*{\tr}{tr}
\DeclareMathOperator*{\rk}{rank}
\DeclareMathOperator*{\im}{im}
\DeclareMathOperator*{\epi}{epi}
% \char command is already defined
\DeclareMathOperator*{\fchar}{char}


\theoremstyle{definition}
\newtheorem{mythm}{Theorem}
\newtheorem{mylem}{Lemma}
\newtheorem{mycor}{Corollary}
\newtheorem{myprop}{Proposition}
\newtheorem{mydef}{Definition}

\newcommand{\yaxis}{$y$\nobreakdash-axis\xspace}
\newcommand{\xaxis}{$x$\nobreakdash-axis\xspace}

\newcommand{\algname}[1]{{\textsc{#1}}\xspace}

% Hera stuff

\newcommand{\ignore}[1]{}

\newcommand{\dtype}[1]{{\textit{\small #1}}}

\newcommand{\libraryname}[1]{{\textsc{#1}}\xspace}

\newcommand{\hera}{\libraryname{Hera}}
\newcommand{\Hera}{\libraryname{Hera}}
\newcommand{\dionisus}{\libraryname{Dionysus}}
\newcommand{\Dionisus}{\libraryname{Dionysus}}

%\numberwithin{equation}{section}
%\numberwithin{figure}{section}

%\newcommand{\Remark}[1]     {{\small {\sf [[#1]]}}}

%%% pgfplots %%%
%\tikzexternalize

\pgfplotsset
{
    tick label style={font=\footnotesize},
    legend style    ={font=\footnotesize},
    label style     ={font=\footnotesize},
    tickwidth       =2pt,
    ylabel near ticks,
    xlabel near ticks,
    legend style     ={draw=none},
    legend cell align=right,
    legend plot pos  =right,
    /pgfplots/narrow area legend/.style=
    {
        legend image code/.code={ \draw[#1] (0cm,-0.06cm) rectangle (0.6cm,0.06cm); }
    },
    /pgfplots/short line legend/.style=
    {
        legend image code/.code={ \draw[#1] (0cm,0cm) rectangle (0.6cm,0cm); }
    },
}

\pgfplotscreateplotcyclelist{filled-color}
{
    every mark/.append style={solid,fill=blue!80!white},    mark=*          \\%
    every mark/.append style={solid,fill=red!80!white},     mark=square*    \\%
    every mark/.append style={solid,fill=orange!80!white},  mark=pentagon*  \\%
    every mark/.append style={solid,fill=green},            mark=diamond*   \\%
    every mark/.append style={solid,fill=black},            mark=triangle*  \\%
}

\newcommand{\addaveraged}[2][]
           {\addplot+[#1,error bars/.cd, y dir=both,y explicit] table[x index=0, y index=1, y error index=2] {#2};}

\pgfplotsset{
    geommark/.style={black,mark options={solid,fill=red!80!white},mark=square*},
    nongeommark/.style={black,mark options={solid,fill=blue!80!white},mark=*},
    dionmark/.style={black,mark options={solid,fill=orange!80!white},mark=pentagon*},
    greentrianglemark/.style={black,mark options={solid,fill=green!80!white},mark=triangle*},
    browndiamondmark/.style={black,mark options={solid,fill=brown!80!white},mark=diamond*}
}


% all distances
% interleaving
\newcommand{\intdist}{{D_I}}
% bottleneck
\newcommand{\btdist}{{W_\infty}}
% Wasserstein of order q (parameter)
\newcommand{\wsdist}[1]{{W_{#1}}}

% expensive distances stuff
\def \expDistFigPath {data/expensive_distances/pics/}
\def \expDistDataPath {data/expensive_distances/}
\newcommand{\ra}[1]{\renewcommand{\arraystretch}{#1}}

\renewcommand{\S}{\mathbb{S}}
\newcommand{\dappr}{\tilde{d}}
\newcommand{\orient}{O}
\newcommand{\cpro}{\Pi}

\newcommand{\metricspace}{\mathcal{M}}
\newcommand{\approxmetric}{\gamma}
\newcommand{\parent}{\mathrm{par}}
\newcommand{\point}{p}
\newcommand{\level}{\ell}
\newcommand{\diam}{\mathrm{diam}}
\newcommand{\diag}{{\Delta}}
\newcommand{\pointset}{P}
\newcommand{\distspace}{\mathcal{M}}
\newcommand{\dist}{\delta}
\newcommand{\adist}{\gamma}
\newcommand{\complexity}{C_{\dist}}
\newcommand{\doublingdimension}{\Delta}


\newcommand{\espanner}{{$(1+\eps)$-spanner}}

\newcommand{\grdy}{\algname{Greedy}}
\newcommand{\bgrdy}{\algname{BlindGreedy}}
\newcommand{\brndm}{\algname{BlindRandom}}
\newcommand{\bqsgrdy}{\algname{BlindQuasiSortedGreedy}}
\newcommand{\bqsshaker}{\algname{BlindQuasiSortedShaker}}

\newcommand{\tX}{{\tilde{X}}}
\newcommand{\tY}{{\tilde{Y}}}

\newcommand{\myparagraph}[1]{\textbf{#1.}}

\def\marrow{\marginpar[\hfill$\longrightarrow$]{$\longleftarrow$}}
\def\michael#1{\textcolor{red}{\textsc{Michael says: }{\marrow\sf #1}}}




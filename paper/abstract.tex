\documentclass[a4paper,USenglish]{socg-lipics-v2018}
\usepackage{amsthm}
\usepackage{amsmath}
\usepackage{mathbbol}
\usepackage{amsfonts}
\usepackage{amssymb}
\usepackage{graphicx}
\usepackage{algorithm}
\usepackage{url}
%\usepackage{floatflt}
%\usepackage{mathptmx}
\usepackage{epsfig}
\usepackage{wrapfig}
\usepackage{color}

\usepackage{algorithm}
\usepackage{algpseudocode}
\usepackage{comment}
\usepackage{braket}

% \usepackage{ccaption}

\newcommand{\N}{\mathbb{N}}
\newcommand{\Q}{\mathbb{Q}}
\newcommand{\R}{\mathbb{R}}
\newcommand{\Z}{\mathbb{Z}}
\renewcommand{\S}{\mathbb{S}}
\newcommand{\dappr}{\tilde{d}}
%\newcommand{\Wlog}{W.l.o.g.\ }
%\newcommand{\wlog}{w.l.o.g.\ }
\newcommand{\eps}{\epsilon}
\newcommand{\orient}{O}
\newcommand{\cpro}{\Pi}
\newcommand{\fib}{\mathrm{Fib}}

\newcommand{\parent}{\mathrm{par}}
\newcommand{\point}{p}
\newcommand{\level}{\ell}
\newcommand{\diam}{\mathrm{diam}}
\newcommand{\pointset}{P}
\newcommand{\distspace}{\mathcal{M}}
\newcommand{\dist}{\delta}
\newcommand{\adist}{\gamma}
\newcommand{\complexity}{C_{\dist}}
\newcommand{\doublingdimension}{\Delta}

%\newcommand{\remark}[1]{\textbf{[#1]}}

%\newtheorem{theorem}{Theorem}
%\newdef{definition}{Definition}
%\newtheorem{lemma}[theorem]{Lemma}
%\newtheorem{corollary}[theorem]{Corollary}

\newcommand{\myparagraph}[1]{\textbf{#1.}}

%\newtheorem{theorem}{Theorem}
%\newtheorem{lemma}[theorem]{Lemma}
%\newtheorem{proposition}[theorem]{Proposition}
%\newtheorem{corollary}[theorem]{Corollary}
%\newtheorem{definition}[theorem]{Definition}
%\newtheorem{remark}[theorem]{Remark}
%\newtheorem{note}[theorem]{Note}

%\numberwithin{equation}{section}
%\numberwithin{figure}{section}

%\def\marrow{\marginpar[\hfill$\longrightarrow$]{$\longleftarrow$}}
%\def\michael#1{\textcolor{red}{\textsc{Michael says: }{\marrow\sf #1}}}

\title{Geometric Filters for Metric Spaces}
\author{Michael Kerber}{TU Graz}{kerber@tugraz.at}{}{}
\author{Arnur Nigmetov}{TU Graz}{nigmetov@tugraz.at}{}{}

\begin{document}
\maketitle

\section{Abstract}
Often the complexity bounds for algorithms in metric spaces
assume that the metric can be computed in constant time and
subsume this constant in $O()$ notation. 
Motivated by examples of metrics from Topological Data Analysis,
in this paper we consider
the case of metric spaces where each distance computation
is an expensive operation.
We consider two classical problems: computing a spanner and finding approximate
nearest neighbor in the setting when the number of input points $n$
is small enough to consider operations with polynomial complexity in $n$ (even of
relatively high degree) as something cheap compared to the computation of an individual distance,
and aim to reduce the number of computed distances.
An example of such a situation is having about a hundred of persistence
diagrams, each containing hundreds of thousands of points.



\remark{start of the short version}

We propose simple algorithms  for each of the two problems,
and for the approximate nearest neighbor we are able to 
prove theoretically that the algorithm
reduces the number of computed distances to $O(\log n)$,
if we assume doubling metric space.
We also provide experimental evidence supporting this bound.


For spanners we can only provide experimental results for
different heuristics that we tried; but the results look
convincing in the sense that on many different random point
sets we are able to compute the spanner 
avoiding computation of all $n \choose 2$ pairwise distances.


The main idea of our approach to both problems is to use
the triangle inequality to extract as much information as possible
from the distances that we have computed so far, and
it turns out that in case of low doubling dimension
this idea really works, sparing us from the computation
of many distances we would have needed, if we
chose some of the existing algorithms. To the best of our knowledge,
this is the first paper in which this setting is considered,
and we believe that these results are just first
steps in this direction, which is theoretically
appealing as well as practically important.



\remark{end of the short version, start of the long version}


The main idea of our approach to the both problems is to maintain
lower and upper bounds
for each of the distances we might need to compute
and use these bounds whenever possible instead
of the true distance. For example, when the algorithm
needs to decide whether the distance between two given points
is less than some $t$, and the upper bound for this distance is
less than $t$ or the lower bound is larger than $t$,
we avoid computing the actual distance;
otherwise we are forced to compute it, but then
we use it to update our bounds for all other distances (the
only tool that we can use here is, of course, the triangle inequality).
For spanners we propose what we call the \textit{blind} approach,
where we add an edge to the spanner for each of the distances we computed 
(hence the name: the
algorithm has no access to exact distance between two points before it decides to connect them
with an edge), and we continue adding edges until the lower and upper bounds
for all pairs of points become sufficiently close to each other.
We experimentally demonstrate that in case of low dimension
some of the heuristics we tried help to significantly reduce the number of
computed distances.


For nearest neighbor problem we want to minimize the number
of computed distances to the query point. Here we are able to theoretically prove that, 
if we assume doubling property,
it is possible to find the approximate nearest neighbor computing
only $O(\log n)$ distances to the query point.
We also provide experimental evidence showing that this approach
reduces the number of distance computations in practice.


As far as we know, the setting that we consider is new, and the results in this paper
are just the first steps in this direction, which, we believe, has large
theoretical potential and is of great practical importance.

\end{document}

